\documentclass{article}

\usepackage[german]{babel}
\usepackage{tikz}
\usetikzlibrary{positioning}

\begin{document}
\section{Einleitung}

\section{Daten}
\subsection{Die Creditreform Datenbank}
\subsubsection{Datenaufbereitung}

\section{Methoden}
\subsection{Neuronale Netze}
\subsection{Theorie}
\subsection{Architekturen und Hyperparameter}

\tikzset{
  every neuron/.style={
    circle,
    draw,
    minimum size=1cm},
  neuron missing/.style={
    draw=none,
    scale=4,
    text height=0.333cm,
    execute at begin node=\color{black}$\vdots$},
}

\begin{figure}
\centering
\caption{Neuronales Netz mit einer versteckten Schicht.}
\vspace{2mm}
\begin{tikzpicture}[x=1.5cm, y=1.5cm, >=stealth]
% plot input layer
\foreach \m/\1 [count=\y] in {1, 2, missing, 3}
  \node [every neuron/.try, neuron \m/.try] (input-\m) at (0, 2.5-\y) {};

\foreach \m [count=\y] in {1, missing, 2}
  \node [every neuron/.try, neuron \m/.try] (hidden-\m) at (2, 2-\y*1.1) {};

\foreach \m [count=\y] in {1}
  \node [every neuron/.try, neuron \m/.try] (output-\m) at (4, 0) {};

\foreach \l [count=\i] in {1,2,28}
  \draw [<-] (input-\i) -- ++(-1,0)
    node [above, midway] {$X_{\l}$};

\foreach \l [count=\i] in {1}
  \draw [->] (output-\i) -- ++ (1,0)
    node [above, midway] {$O_\i$};

\foreach \l [count=\i] in {1, 10}
  \node [above] at (hidden-\i.north) {$H_\l$};
  
\foreach \i in {1, 2, 3}
  \foreach \j in {1,2}
    \draw [->] (input-\i) -- (hidden-\j);

\foreach \i in {1, 2}
  \draw [->] (hidden-\i) -- (output-1);


\foreach \l [count=\x from 0] in {Input-, Versteckte, Output-}
  \node [align=center, above] at (\x*2,2) {\l \\ Schicht};
\end{tikzpicture}
\end{figure}


\begin{figure}
\centering
\caption{Neuronales Netz mit zwei versteckten Schichten.}
\vspace{2mm}
\begin{tikzpicture}[x=1.5cm, y=1.5cm, >=stealth]
\foreach \m/\1 [count=\y] in {1, 2, missing, 3}
  \node [every neuron/.try, neuron \m/.try] (input-\m) at (0, 2.5-\y) {};

\foreach \m [count=\y] in {1, missing, 2}
  \node [every neuron/.try, neuron \m/.try] (hidden1-\m) at (2, 2-\y*1.1) {};

\foreach \m [count=\y] in {1, missing, 2}
  \node [every neuron/.try, neuron \m/.try] (hidden2-\m) at (4, 2-\y*1.1) {};

\foreach \m [count=\y] in {1}
  \node [every neuron/.try, neuron \m/.try] (output-\m) at (6, 0) {};

\foreach \l [count=\i] in {1,2,28}
  \draw [<-] (input-\i) -- ++(-1,0)
    node [above, midway] {$X_{\l}$};

\foreach \l [count=\i] in {1}
  \draw [->] (output-\i) -- ++ (1,0)
    node [above, midway] {$O_\i$};

\foreach \l [count=\i] in {1, 10}
  \node [above] at (hidden1-\i.north) {$H_{1\l}$};

\foreach \l [count=\i] in {1, 10}
  \node [above] at (hidden2-\i.north) {$H_{2\l}$};

\foreach \i in {1, 2}
  \foreach \j in {1,2}
    \draw [->] (hidden1-\i) -- (hidden2-\j);

\foreach \i in {1, 2}
  \draw [->] (hidden2-\i) -- (output-1);
  
\foreach \i in {1, 2, 3}
  \foreach \j in {1,2}
    \draw [->] (input-\i) -- (hidden1-\j);

\foreach \l [count=\x from 0] in {Input-, Erste Versteckte, Zweite Versteckte, Ouput-}
  \node [align=center, above] at (\x*2,2) {\l \\ Schicht};
\end{tikzpicture}
\end{figure}


\begin{figure}
\centering
\caption{Neuronales Netz mit fünf versteckten Schichten.}
\vspace{2mm}
\begin{tikzpicture}[x=1.5cm, y=1.5cm, >=stealth]
% input
\foreach \m/\1 [count=\y] in {1, 2, missing, 3}
  \node [every neuron/.try, neuron \m/.try] (input-\m) at (0, 2.5-\y) {};

\foreach \m [count=\y] in {1, missing, 2}
  \node [every neuron/.try, neuron \m/.try] (hidden1-\m) at (1, 2-\y*1.1) {};

\foreach \m [count=\y] in {1, missing, 2}
  \node [every neuron/.try, neuron \m/.try] (hidden2-\m) at (2, 2-\y*1.1) {};

\foreach \m [count=\y] in {1, missing, 2}
  \node [every neuron/.try, neuron \m/.try] (hidden3-\m) at (3, 2-\y*1.1) {};

\foreach \m [count=\y] in {1, missing, 2}
  \node [every neuron/.try, neuron \m/.try] (hidden4-\m) at (4, 2-\y*1.1) {};

\foreach \m [count=\y] in {1, missing, 2}
  \node [every neuron/.try, neuron \m/.try] (hidden5-\m) at (5, 2-\y*1.1) {};


\foreach \m [count=\y] in {1}
  \node [every neuron/.try, neuron \m/.try] (output-\m) at (6, 0) {};

\foreach \l [count=\i] in {1,2,28}
  \draw [<-] (input-\i) -- ++(-1,0)
    node [above, midway] {$X_{\l}$};

\foreach \l [count=\i] in {1}
  \draw [->] (output-\i) -- ++ (1,0)
    node [above, midway] {$O_\i$};

% names of hidden units
\foreach \j in {1, 2, 3, 4, 5}
  \foreach \l [count=\i] in {1, 10}
    \node [above] at (hidden\j-\i.north) {$H_{\j\l}$};

% edges between hidden units
\foreach \i in {1, 2}
  \foreach \j in {1,2}
    \draw [->] (hidden1-\i) -- (hidden2-\j);

\foreach \i in {1, 2}
  \foreach \j in {1,2}
    \draw [->] (hidden2-\i) -- (hidden3-\j);
    
\foreach \i in {1, 2}
  \foreach \j in {1,2}
    \draw [->] (hidden3-\i) -- (hidden4-\j);

\foreach \i in {1, 2}
  \foreach \j in {1,2}
    \draw [->] (hidden4-\i) -- (hidden5-\j);

% last edge from 5th hidden layer to output
\foreach \i in {1, 2}
  \draw [->] (hidden5-\i) -- (output-1);
  
\foreach \i in {1, 2, 3}
  \foreach \j in {1,2}
    \draw [->] (input-\i) -- (hidden1-\j);

\foreach \l [count=\x from 0] in {Input-, Versteckte, Ouput-}
  \node [align=center, above] at (\x*3,2) {\l \\ Schicht(en)};
\end{tikzpicture}
\end{figure}


\subsection{Ergebnisse}

\section{Zusammenfassung}
\end{document}